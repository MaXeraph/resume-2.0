% YAAC Another Awesome CV LaTeX Template
%
% This template has been downloaded from:
% https://github.com/darwiin/yaac-another-awesome-cv
%
% Author:
% Christophe Roger
%
% Template license:
% CC BY-SA 4.0 (https://creativecommons.org/licenses/by-sa/4.0/)
%Section: Work Experience at the top
\sectionTitle{Professional Experience}{\faSuitcase}
%\renewcommand{\labelitemi}{$\bullet$}
\begin{experiences}
  \experienceCurrent
    {Software Engineer I | Platform}{\href{https://uken.com}{Uken Games}}{Toronto}
    {July 2022} {
                    \begin{itemize}
                        \setlength\itemsep{0.2em}
                        \item Increasing IAP revenue by \textbf{25\%} by integrating new payment system with Xsolla 

                        \item Increased marketing performance and cost reduction with In-house Mobile Dynamic Messaging System.
                        
                        \item Reduced AWS MSK deployment maintainance time by placing Broker URLs behind DNS

                        \item Enabling better visibility and customizability by rebuilding custom Github Pull Request notifier bot with Python. Deployed with AWS SAM

                        \item Enhanced community support by implementing a Discord notifier bot for a Live Service game
                    \end{itemize}
                    }
                    {Reactive Spring, Web3 Manifold, AWS, Python, Kafka, Git LFS}

  \experienceCurrent  
    {Software Engineer I | Platform}{\href{https://uken.com}{Uken Games}}{Toronto}
    {July 2022} {
                    \begin{itemize}
                        \setlength\itemsep{0.2em}
                        \item Increasing IAP revenue by \textbf{25\%} by integrating new payment system with Xsolla 

                        \item Increased marketing performance and cost reduction with In-house Mobile Dynamic Messaging System.
                        
                        \item Reduced AWS MSK deployment maintainance time by placing Broker URLs behind DNS

                        \item Enabling better visibility and customizability by rebuilding custom Github Pull Request notifier bot with Python. Deployed with AWS SAM

                        \item Enhanced community support by implementing a Discord notifier bot for a Live Service game
                    \end{itemize}
                    }
                    {Reactive Spring, Web3 Manifold, AWS, Python, Kafka, Git LFS}
                    
  \experience
    {Oct 2021}   {Software Engineer PEY | Platform}{\href{https://uken.com}{Uken Games}}{Toronto}
    {May 2020} {
                    \begin{itemize}
                        \setlength\itemsep{0.2em}
                        \item Reduced \textbf{100\%} cost with Slack bot incident communication by moving to \textbf{AWS SAM} from EC2
                        
                        \item Reduced \textbf{30\%} unproductive commits by enforcing codestyle using \link{https://prettier.io/}{\underline{Prettier} } plugin through Jenkins
                        
                        \item Managed \textbf{1000+} Github resources by designing \textbf{Terraform-based} system with changes through \textbf{JSON} format

                        \item Increased marketing's outreach by \textbf{28\%} with API scheduling system based of \link{https://github.com/kagkarlsson/db-scheduler}{\underline{db-scheduler} } library
                        
                        \item Consolidated \textbf{100\%} of technical documentation to a single Wiki solution: \link{https://js.wiki/}{\underline{Wiki.JS}}. Deployed to \textbf{AWS} with authentication using \link{https://www.onelogin.com/}{\underline{OneLogin SAML}}
                    \end{itemize}
                    }
                    {Terraform, AWS, Python, Docker, IntelliJ Idea, Spring Boot, Jenkins}
\end{experiences}
